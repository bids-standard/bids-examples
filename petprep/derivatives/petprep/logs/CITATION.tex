% Options for packages loaded elsewhere
\PassOptionsToPackage{unicode}{hyperref}
\PassOptionsToPackage{hyphens}{url}
\documentclass[
]{article}
\usepackage{xcolor}
\usepackage{amsmath,amssymb}
\setcounter{secnumdepth}{-\maxdimen} % remove section numbering
\usepackage{iftex}
\ifPDFTeX
  \usepackage[T1]{fontenc}
  \usepackage[utf8]{inputenc}
  \usepackage{textcomp} % provide euro and other symbols
\else % if luatex or xetex
  \usepackage{unicode-math} % this also loads fontspec
  \defaultfontfeatures{Scale=MatchLowercase}
  \defaultfontfeatures[\rmfamily]{Ligatures=TeX,Scale=1}
\fi
\usepackage{lmodern}
\ifPDFTeX\else
  % xetex/luatex font selection
\fi
% Use upquote if available, for straight quotes in verbatim environments
\IfFileExists{upquote.sty}{\usepackage{upquote}}{}
\IfFileExists{microtype.sty}{% use microtype if available
  \usepackage[]{microtype}
  \UseMicrotypeSet[protrusion]{basicmath} % disable protrusion for tt fonts
}{}
\makeatletter
\@ifundefined{KOMAClassName}{% if non-KOMA class
  \IfFileExists{parskip.sty}{%
    \usepackage{parskip}
  }{% else
    \setlength{\parindent}{0pt}
    \setlength{\parskip}{6pt plus 2pt minus 1pt}}
}{% if KOMA class
  \KOMAoptions{parskip=half}}
\makeatother
\setlength{\emergencystretch}{3em} % prevent overfull lines
\providecommand{\tightlist}{%
  \setlength{\itemsep}{0pt}\setlength{\parskip}{0pt}}
\usepackage[]{natbib}
\bibliographystyle{plainnat}
\usepackage{bookmark}
\IfFileExists{xurl.sty}{\usepackage{xurl}}{} % add URL line breaks if available
\urlstyle{same}
\hypersetup{
  hidelinks,
  pdfcreator={LaTeX via pandoc}}

\author{}
\date{}

\begin{document}

Results included in this manuscript come from preprocessing performed
using \emph{PETPrep} 0.0.3 (\citet{fmriprep1}; \citet{fmriprep2};
RRID:SCR\_016216), which is based on \emph{Nipype} 1.9.2
(\citet{nipype1}; \citet{nipype2}; RRID:SCR\_002502).

\begin{description}
\item[Anatomical data preprocessing]
A total of 1 T1-weighted (T1w) images were found within the input BIDS
dataset. The T1w image was corrected for intensity non-uniformity (INU)
with \texttt{N4BiasFieldCorrection} \citep{n4}, distributed with ANTs
2.6.2 \citep[RRID:SCR\_004757]{ants}, and used as T1w-reference
throughout the workflow. The T1w-reference was then skull-stripped with
a \emph{Nipype} implementation of the \texttt{antsBrainExtraction.sh}
workflow (from ANTs), using OASIS30ANTs as target template. Brain tissue
segmentation of cerebrospinal fluid (CSF), white-matter (WM) and
gray-matter (GM) was performed on the brain-extracted T1w using
\texttt{fast} \citep[FSL (version unknown),
RRID:SCR\_002823,][]{fsl_fast}. Brain surfaces were reconstructed using
\texttt{recon-all} \citep[FreeSurfer 7.4.1,
RRID:SCR\_001847,][]{fs_reconall}, and the brain mask estimated
previously was refined with a custom variation of the method to
reconcile ANTs-derived and FreeSurfer-derived segmentations of the
cortical gray-matter of Mindboggle
\citep[RRID:SCR\_002438,][]{mindboggle}. Volume-based spatial
normalization to one standard space (MNI152NLin2009cAsym) was performed
through nonlinear registration with \texttt{antsRegistration} (ANTs
2.6.2), using brain-extracted versions of both T1w reference and the T1w
template. The following template was were selected for spatial
normalization and accessed with \emph{TemplateFlow}
\citep[24.2.2,][]{templateflow}: \emph{ICBM 152 Nonlinear Asymmetrical
template version 2009c} {[}\citet{mni152nlin2009casym},
RRID:SCR\_008796; TemplateFlow ID: MNI152NLin2009cAsym{]}.
\item[PET data preprocessing]
For each of the 2 PET runs found per subject (across all tasks and
sessions), the following preprocessing steps were performed. Robust head
motion estimation and correction were carried out after generating a
reference image, which was subsequently coregistered to the T1-weighted
anatomical image. A brain mask was computed and the structural image was
segmented with the \texttt{gtm} segmentation workflow from FreeSurfer.
Head-motion parameters with respect to the PET reference (transformation
matrices, and six corresponding rotation and translation parameters) are
estimated before any spatiotemporal filtering using FreeSurfer's
\texttt{mri\_robust\_template}. Several confounding time-series were
calculated based on the \emph{preprocessed PET}: framewise displacement
(FD), DVARS and three region-wise global signals. FD was computed using
two formulations following Power (absolute sum of relative motions,
\citet{power_fd_dvars}) and Jenkinson (relative root mean square
displacement between affines, \citet{mcflirt}). FD and DVARS are
calculated for each PET run, both using their implementations in
\emph{Nipype} \citep[following the definitions by][]{power_fd_dvars}.
The three global signals are extracted within the CSF, the WM, and the
whole-brain masks. Additionally, a set of physiological regressors were
extracted to allow for component-based noise correction
\citep[\emph{CompCor},][]{compcor}. Principal components are estimated
after high-pass filtering the \emph{preprocessed PET} time-series (using
a discrete cosine filter with 128s cut-off) for the two \emph{CompCor}
variants: temporal (tCompCor) and anatomical (aCompCor). tCompCor
components are then calculated from the top 2\% variable voxels within
the brain mask. For aCompCor, three probabilistic masks (CSF, WM and
combined CSF+WM) are generated in anatomical space. The implementation
differs from that of Behzadi et al.~in that instead of eroding the masks
by 2 pixels on PET space, a mask of pixels that likely contain a volume
fraction of GM is subtracted from the aCompCor masks. This mask is
obtained by dilating a GM mask extracted from the FreeSurfer's
\emph{aseg} segmentation, and it ensures components are not extracted
from voxels containing a minimal fraction of GM. Finally, these masks
are resampled into PET space and binarized by thresholding at 0.99 (as
in the original implementation). Components are also calculated
separately within the WM and CSF masks. For each CompCor decomposition,
the \emph{k} components with the largest singular values are retained,
such that the retained components' time series are sufficient to explain
50 percent of variance across the nuisance mask (CSF, WM, combined, or
temporal). The remaining components are dropped from consideration. The
head-motion estimates calculated in the correction step were also placed
within the corresponding confounds file. The confound time series
derived from head motion estimates and global signals were expanded with
the inclusion of temporal derivatives and quadratic terms for each
\citep{confounds_satterthwaite_2013}. Frames that exceeded a threshold
of 0.5 mm FD or 1.5 standardized DVARS were annotated as motion
outliers. Additional nuisance timeseries are calculated by means of
principal components analysis of the signal found within a thin band
(\emph{crown}) of voxels around the edge of the brain, as proposed by
\citep{patriat_improved_2017}. All resamplings can be performed with
\emph{a single interpolation step} by composing all the pertinent
transformations (i.e.~head-motion transform matrices, susceptibility
distortion correction when available, and co-registrations to anatomical
and output spaces). Gridded (volumetric) resamplings were performed
using \texttt{nitransforms}, configured with cubic B-spline
interpolation.
\end{description}

Many internal operations of \emph{PETPrep} use \emph{Nilearn} 0.11.1
\citep[RRID:SCR\_001362]{nilearn}, mostly within the PET processing
workflow. For more details of the pipeline, see
\href{https://petprep.readthedocs.io/en/latest/workflows.html}{the
section corresponding to workflows in \emph{PETPrep}'s documentation}.

\subsubsection{Copyright Waiver}\label{copyright-waiver}

The above boilerplate text was automatically generated by PETPrep with
the express intention that users should copy and paste this text into
their manuscripts \emph{unchanged}. It is released under the
\href{https://creativecommons.org/publicdomain/zero/1.0/}{CC0} license.

\subsubsection{References}\label{references}

Results included in this manuscript come from preprocessing performed
using \emph{PETPrep} 0.0.3 (\citet{fmriprep1}; \citet{fmriprep2};
RRID:SCR\_016216), which is based on \emph{Nipype} 1.9.2
(\citet{nipype1}; \citet{nipype2}; RRID:SCR\_002502).

\begin{description}
\item[Anatomical data preprocessing]
A total of 1 T1-weighted (T1w) images were found within the input BIDS
dataset. The T1w image was corrected for intensity non-uniformity (INU)
with \texttt{N4BiasFieldCorrection} \citep{n4}, distributed with ANTs
2.6.2 \citep[RRID:SCR\_004757]{ants}, and used as T1w-reference
throughout the workflow. The T1w-reference was then skull-stripped with
a \emph{Nipype} implementation of the \texttt{antsBrainExtraction.sh}
workflow (from ANTs), using OASIS30ANTs as target template. Brain tissue
segmentation of cerebrospinal fluid (CSF), white-matter (WM) and
gray-matter (GM) was performed on the brain-extracted T1w using
\texttt{fast} \citep[FSL (version unknown),
RRID:SCR\_002823,][]{fsl_fast}. Brain surfaces were reconstructed using
\texttt{recon-all} \citep[FreeSurfer 7.4.1,
RRID:SCR\_001847,][]{fs_reconall}, and the brain mask estimated
previously was refined with a custom variation of the method to
reconcile ANTs-derived and FreeSurfer-derived segmentations of the
cortical gray-matter of Mindboggle
\citep[RRID:SCR\_002438,][]{mindboggle}. Volume-based spatial
normalization to one standard space (MNI152NLin2009cAsym) was performed
through nonlinear registration with \texttt{antsRegistration} (ANTs
2.6.2), using brain-extracted versions of both T1w reference and the T1w
template. The following template was were selected for spatial
normalization and accessed with \emph{TemplateFlow}
\citep[24.2.2,][]{templateflow}: \emph{ICBM 152 Nonlinear Asymmetrical
template version 2009c} {[}\citet{mni152nlin2009casym},
RRID:SCR\_008796; TemplateFlow ID: MNI152NLin2009cAsym{]}.
\item[PET data preprocessing]
For each of the 1 PET runs found per subject (across all tasks and
sessions), the following preprocessing steps were performed. Robust head
motion estimation and correction were carried out after generating a
reference image, which was subsequently coregistered to the T1-weighted
anatomical image. A brain mask was computed and the structural image was
segmented with the \texttt{gtm} segmentation workflow from FreeSurfer.
Head-motion parameters with respect to the PET reference (transformation
matrices, and six corresponding rotation and translation parameters) are
estimated before any spatiotemporal filtering using FreeSurfer's
\texttt{mri\_robust\_template}. Several confounding time-series were
calculated based on the \emph{preprocessed PET}: framewise displacement
(FD), DVARS and three region-wise global signals. FD was computed using
two formulations following Power (absolute sum of relative motions,
\citet{power_fd_dvars}) and Jenkinson (relative root mean square
displacement between affines, \citet{mcflirt}). FD and DVARS are
calculated for each PET run, both using their implementations in
\emph{Nipype} \citep[following the definitions by][]{power_fd_dvars}.
The three global signals are extracted within the CSF, the WM, and the
whole-brain masks. Additionally, a set of physiological regressors were
extracted to allow for component-based noise correction
\citep[\emph{CompCor},][]{compcor}. Principal components are estimated
after high-pass filtering the \emph{preprocessed PET} time-series (using
a discrete cosine filter with 128s cut-off) for the two \emph{CompCor}
variants: temporal (tCompCor) and anatomical (aCompCor). tCompCor
components are then calculated from the top 2\% variable voxels within
the brain mask. For aCompCor, three probabilistic masks (CSF, WM and
combined CSF+WM) are generated in anatomical space. The implementation
differs from that of Behzadi et al.~in that instead of eroding the masks
by 2 pixels on PET space, a mask of pixels that likely contain a volume
fraction of GM is subtracted from the aCompCor masks. This mask is
obtained by dilating a GM mask extracted from the FreeSurfer's
\emph{aseg} segmentation, and it ensures components are not extracted
from voxels containing a minimal fraction of GM. Finally, these masks
are resampled into PET space and binarized by thresholding at 0.99 (as
in the original implementation). Components are also calculated
separately within the WM and CSF masks. For each CompCor decomposition,
the \emph{k} components with the largest singular values are retained,
such that the retained components' time series are sufficient to explain
50 percent of variance across the nuisance mask (CSF, WM, combined, or
temporal). The remaining components are dropped from consideration. The
head-motion estimates calculated in the correction step were also placed
within the corresponding confounds file. The confound time series
derived from head motion estimates and global signals were expanded with
the inclusion of temporal derivatives and quadratic terms for each
\citep{confounds_satterthwaite_2013}. Frames that exceeded a threshold
of 0.5 mm FD or 1.5 standardized DVARS were annotated as motion
outliers. Additional nuisance timeseries are calculated by means of
principal components analysis of the signal found within a thin band
(\emph{crown}) of voxels around the edge of the brain, as proposed by
\citep{patriat_improved_2017}. All resamplings can be performed with
\emph{a single interpolation step} by composing all the pertinent
transformations (i.e.~head-motion transform matrices, susceptibility
distortion correction when available, and co-registrations to anatomical
and output spaces). Gridded (volumetric) resamplings were performed
using \texttt{nitransforms}, configured with cubic B-spline
interpolation.
\end{description}

Many internal operations of \emph{PETPrep} use \emph{Nilearn} 0.11.1
\citep[RRID:SCR\_001362]{nilearn}, mostly within the PET processing
workflow. For more details of the pipeline, see
\href{https://petprep.readthedocs.io/en/latest/workflows.html}{the
section corresponding to workflows in \emph{PETPrep}'s documentation}.

\subsubsection{Copyright Waiver}\label{copyright-waiver-1}

The above boilerplate text was automatically generated by PETPrep with
the express intention that users should copy and paste this text into
their manuscripts \emph{unchanged}. It is released under the
\href{https://creativecommons.org/publicdomain/zero/1.0/}{CC0} license.

\subsubsection{References}\label{references-1}

\bibliography{/out/logs/CITATION.bib}

\end{document}
